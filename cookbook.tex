% !TeX spellcheck=en_US

\documentclass[]{article}

\usepackage[contents, index]{cuisine}
\usepackage[a4paper, margin=1.5in, centering]{geometry}

%opening
\title{A Collection of Recipes}
\author{jam}

\begin{document}

\renewcommand*{\recipetitlefont}{\large\bfseries\sffamily}
\renewcommand*{\recipenumberfont}{\large\bfseries\sffamily}
\renewcommand*{\recipequantityfont}{\sffamily\bfseries}
\renewcommand*{\recipeunitfont}{\sffamily}
\renewcommand*{\recipeingredientfont}{\sffamily}
\renewcommand*{\recipefreeformfont}{\itshape}
%\renewcommand*{\recipestepnumberfont}{\Roman}
\maketitle

\tableofcontents
\begin{center}
\begin{recipe}{Pancakes}{2 portions}{\fr12 hour}
	\ingredient[1]{cup}{flour}
	\ingredient[1]{touch}{salt}
	\ingredient[1]{tbsp}{baking powder}
	\ingredient[1]{pack}{vanilla sugar}
	Mix all of the dry ingredients thoroughly in a bowl.
	\ingredient[1]{}{banana}
	\ingredient[1]{cup}{milk}
	Mash the banana into pulp.
	Then mix the banana pulp and milk with the dry ingredients.
	Let the batter rest until some bubbles start to form.
	\ingredient[]{\hspace{11pt}Many}{fruits}
	Optionally, add fruits to the batter.
	Blueberries are recommended.
	\freeform Heat a pan, add margarine or butter and fry pancakes once per side until bubbles start to form.
	\freeform\hrulefill
\end{recipe}

\begin{recipe}{Chocolate Coconut Cereal}{A weeks worth}{1 hour}
	\ingredient[500]{grams}{oats}
	\ingredient[180]{grams}{coconut flakes}
	\ingredient[1]{touch}{salt}
	\ingredient[80]{grams}{cocoa powder}
	Preheat oven to 180°C.
	Thoroughly mix dry ingredients.
	\ingredient[6]{tbsp}{coconut oil}
	\ingredient[100]{grams}{honey}
	Heat coconut oil and honey in a pot and mix.
	Pour mixture over dry ingredients while stirring.
	\newstep
	Spread over baking paper and bake for 10-12 minutes.
	\ingredient[50]{grams}{dark chocolate}
	Sprinkle dark chocolate pieces over the cereal fresh out of the oven.
	Mix and let cool.
	\freeform\hrulefill
\end{recipe}

\begin{recipe}{Lentil Dal}{5 portions}{45 mins}
	\ingredient[2]{}{white onions}
	\ingredient[50]{grams}{ginger}
	\ingredient[5]{cloves}{garlic}
	Chop it all up and fry with oil in a pot.
	\ingredient[3]{tsp}{garam masala}
	Add garam masala and continue to roast briefly.
	\ingredient[500]{grams}{lentils}
	\ingredient[1]{can}{canned tomatoes}
	\ingredient[200]{grams}{fresh tomatoes}
	\ingredient[1]{can}{coconut milk}
	\ingredient{to taste}{salt, pepper, chili}
	Add all to the pot and stir.
	Let simmer for about 20 minutes.
	\freeform Serve with bread, rice or potatoes.
	\freeform\hrulefill
\end{recipe}

\begin{recipe}{Coconut Zucchini Risotto}{3 portions}{1 hour}
	\ing[2]{}{white onions}
	\ing[2]{cloves}{garlic}
	\ing[20]{grams}{ginger}
	Heat oil in a pot and add lightly fry for a few minutes.
	\ing[1]{can}{coconut milk}
	\ing[700]{ml}{vegetable broth}
	Heat broth and coconut milk in separate pot on low heat.
	\ing[250]{grams}{risotto rice}
	Add rice to garlic, onion and ginger. 
	Fry briefly until glassy.
	\newstep
	Gradually add more broth, waiting for the rice to absorb it at each step.
	Cook for about 12 minutes.
	\ing[1]{}{zucchini}
	Add cubed zucchini to the pot. 
	Cook for 8 more minutes.
	\ing[2]{tbsp}{sesame oil}
	\ing[2]{tbsp}{soy sauce}
	\ing[2]{tbsp}{rice vinegar}
	\ing[1]{tbsp}{honey/syrup}
	\ing[250]{grams}{smoked tofu}
	Dice tofu and marinade with other ingredients.
	Heat a pan on medium-high heat and fry tofu until crispy.
	\freeform\hrulefill
\end{recipe}

\begin{recipe}{Pickling Liquid}{As needed}{30 mins}
	\ingredient[1]{cup}{white vinegar}
	\ingredient[1]{cup}{water}
	Mix equal parts vinegar and water, bring to a boil.
	\ingredient[3/4]{cup}{sugar}
	Add sugar once boiling.
	\ingredient[2]{tbsp}{salt}
	\ingredient[4]{pieces}{ginger}
	\ingredient[5]{}{black peppercorns}
	Reduce heat, add spices and keep on low heat for 20 minutes.
	\newstep
	\freeform Slice and prepare the vegetables to be pickled. 
	Pour liquid over the vegetables in a container. 
	Let fully cool before screwing on the lid.
	\freeform\hrulefill
\end{recipe}

\begin{recipe}{Banana Bread}{1 loaf}{90 mins}
	\ingredient[300]{g}{flour}
	\ingredient[10]{g}{baking powder}
	\ingredient[1]{touch}{salt}
	Preheat oven to 180°C. Mix the dry ingredients in a bowl.
	\ingredient[3]{}{ripe bananas}
	\ingredient[250]{g}{applesauce}
	\ingredient[135]{ml}{milk}
	Mash up the bananas to a pulp and mix all wet ingredients in a separate bowl before adding to the dry ones.
	\ingredient[150]{g}{walnuts}
	\ingredient[75]{g}{chocolate}
	Roughly chop walnuts and chocolate and add to the batter. Save some of the walnuts for on top.
	\newstep
	\freeform Bake for 60mins. Cover with a lid or foil after 30mins to prevent walnuts from burning. 
	\freeform\hrulefill
\end{recipe}

\begin{recipe}{Vegetable Gyoza Seasoning}{As needed}{}
	\ingredient[2]{tbsp}{soy sauce}
	\ingredient[1]{tbsp}{miso}
	\ingredient[2]{tsp}{toasted sesame oil}
	\ingredient[1]{tsp}{salt}
	\ingredient[1/2]{tsp}{pepper}
	\ingredient[135]{ml}{milk}
	To be mixed into the gyoza filling before folding. 
	\freeform\hrulefill
\end{recipe}

\end{center}
\end{document}
